% 导言区
%\documentclass{article} % article,book,report,letter
\documentclass{ctexart}  %对应上一行的article
\usepackage{ctex}   %命令行使用texdoc ctex 查看文档

\newcommand\degree{^\circ}  %在这里定义新的命令,使用\newcommand,命令名紧随其后,{ } 中写函数定义
%可在命令行界面使用texdoc lshort-zh 查看latex的中文文档
%\title{My First Document} %与下文对比,加上了字体的设置
%\author{Robotsl}
\title{\heiti 我的第一个文档}
\author{\kaishu 廖兴滨}
\date{\today}
% 正文区

\begin{document}  %一个tex文件中只能出现一个document
	\maketitle  %输出标题,如果是letter则注释这句
	Hello World!
	%如果需要换行,则在代码中多加一空行,注释也占一行
	
	Let  $f(x)$ be defined by the formula $f(x) = 3x^2 + x -1$  which is a polynomial of degree 2.
	%$ something $表示数学模式,不单独占一行
	
	%一些注释
	Let  $f(x)$ be defined by the formula $$f(x) = 3x^2 + x -1$$ which is a polynomial of degree 2.中文
	% $$ something $$ 表示数学模式单独占一行,不加处理的话中文不显示,加了\usepackage{ctex}以后就可以显示中文了.
	勾股定理可以用现代语言表述如下:
	
	直角三角形斜边的平方等于两腰的平方和。
	
	可以用符号语言表示为:设直角三角形 $ABC$,其中 $\angle C = 90 \degree$  (与degree函数做对比  $\angle C = 90 ^\circ $ ):
	\begin{equation}  % equation这个环境用于产生一个带编号的公式
		AB^2 = BC^2 + AC ^2
	\end{equation}
\end{document}