\documentclass{ctexart}

\usepackage{amsmath}
\usepackage{amssymb}

\begin{document}
	\begin{gather}
	% gather 和 gather* 环境(可以使用\\换行
	% 带编号
		a + b = b + a \\
		ab ba
	\end{gather}
	
	\begin{gather*}
	% gather 和 gather* 环境(可以使用\\换行
	% 带编号
	a + b = b + a \\
	3 + 5 = 5 + 3 \\
	ab ba
	\end{gather*}
	
	% 在\\前使用\notag 阻止编号
	\begin{gather}
		a + b = b + a \notag \\
		ab ba \notag
	\end{gather}
	
	% align环境和align*环境(用 & 进行对齐)
	% 带编号
	\begin{align}
		x &= t + \cos t + 1 \\
		y &= 2\sin t
	\end{align}
	
	% 不带编号
	\begin{align*}
	x &= t & x &= \cos t & x &= t\\
	y &= 2t & y &= \sin (t + 1) & y &= \sin t
	\end{align*}
	
	% split 环境(对齐采用 align 环境的方式,编号在中间)
	\begin{equation}
		\begin{split}
			\cos 2x &= \cos ^2 x - \sin ^2 x \\ % \\换行
					&= 2\cos ^2 x - 1
		\end{split}
	\end{equation}
	
	% 分段函数
	% 使用 cases 环境,每行公式中使用 & 分隔为两部分
	% 通常表示值和后面的条件
	\begin{equation}
		D(x) = \begin{cases}
			1, & \text{如果 } x \in \mathbb{Q} \\ %mathbb 命令用于输出花体字符 in命令用于输出属于字符
			%在数学模式下,如果想输出中文需要临时切换到文本模式,否则将不予显示
			0, & \text{如果 } x \in \mathbb{R} \setminus\mathbb{Q}
		\end{cases}
	\end{equation}
	
\end{document}