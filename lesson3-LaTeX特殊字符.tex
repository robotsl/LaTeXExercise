% 导言区
\documentclass{article} % article,book,report,letter [10pt]表示十磅,只有10,11,12这三个
\usepackage{ctex}   %命令行使用texdoc ctex 查看文档
\usepackage{xltxtra} %提供了针对XeTeX的改进并且加入了XeTeX的LOGO
\usepackage{texnames}
\usepackage{mflogo}


\title{\heiti 我的第四个文档}
\author{\kaishu 廖兴滨}
\date{\today}
% 正文区

\begin{document}  %一个tex文件中只能出现一个document
	\section{空白符号}
	%空行分段,多个空行等同一个
	%自动缩进,绝对不能使用空格代替
	%英文中多个空格处理为1个空格,中文中空格将被忽h略
	%汉字与其他字符的间距会自动由XeLaTeX处理
	%禁止使用中文全角空格
	Are you wiser than others? definitely no. in some ways, may it is true. What can you achieve ? a luxurious house? a brilliant car? an admirable career? who knows?  %任意多空格都算一个空格
	
	近年来,随着逆向工程和二位重建技术的发展和应用,获取现实世界中物体的三维数据的帆帆发越来越多的关注和研究,definitely no很多研究机构和商业公司都陆续推出了自己的三维重建系统。%中英文之间自动产生空格
	
	% 1em表示当前字体中M的宽度
	a\quad b
	
	% 2em
	a\qquad b
	
	%约为1/6个em
	a\,b a\thinspace b
	
	% 0.5个em
	a\enspace b
	
	% 空格
	a\ b
	
	%硬空格
	a~b
	
	%1pc = 12pt = 4.218mm
	a\kern 1pc b
	
	a\kern -1em b
	
	a\hskip 1em b
	a\hspace{35pt}b
	
	% 占位宽度
	a\hphantom{xyz}b
	
	%弹性长度
	a\hfill b
	
	\section{\LaTeX 控制符}
	\# \$ \% \{ \} \~{} \^{} \textbackslash \&  % /~ /^ 显示03 \~{} \^{} 显示~ ^
	
	
	\section{排版符号}
	\S \P \dag \ddag \copyright \pounds
	
	\section{\TeX 标志符号}
	\TeX{} \LaTeX{} \LaTeXe{} \XeLaTeX
	
	% texnames 宏包提供
	\AmSTeX{} \AmS-\LaTeX{}
	\BibTeX{} \LuaTeX{}
	
	%mflogo 宏包提供
	\METAFONT{} \MF{} \MP{}
	
	\section{引号} %`是1左边的那个键
	` ' `` ''
	
	如 ``你好呢''
	
	\section{连字符}
	- -- ---
	
	\section{非英文字符}
	\oe \OE \ae \AE \aa \AA \o \O \l \L \ss \SS !` ?`
	
	\section{重音符号(以o为例)}
	\`o \'o \^o \''o \~o \=o \.o \u{o} \v{o} \H{o} \r{o} \t{o} \b{o} \c{o} \d{o}
		
\end{document}