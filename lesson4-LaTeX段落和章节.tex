% 导言区
%\documentclass{article} % article,book,report,letter
\documentclass{ctexbook} % 标题居中
%\usepackage{ctex}   
\ctexset{   %独立设置段落title的格式
	section = {
		format+ = \zihao{-4} \heiti \raggedright,
		name = {,、 },
		number = \chinese{section},
		beforeskip = 1.0ex plus 0.2ex minus .2ex,
		afterskip  = 1.0ex plus 0.2ex minus .2ex,
		aftername = \hspace{0pt}
	},
	subsection = {
		format+ = \zihao{5} \heiti \raggedright,
		name = {,、 },
		number = \arabic{subsection},
		beforeskip = 1.0ex plus 0.2ex minus .2ex,
		afterskip  = 1.0ex plus 0.2ex minus .2ex,
		aftername = \hspace{0pt}
	}
	
}



\begin{document}  %一个tex文件中只能出现一个document
	%\maketitle %\maketitle只有article有
	\tableofcontents
	
	\chapter{绪论}report和book类型的排版有chapter,article里没有
	\section{引言小结} % 篇章段落
	环境:win7 + texlive 2016 + TeXstudio
	
	问题:xelatex编译一个很短的文章时非常慢
	
	解决方法:
	
	A. 以管理员身份运行fc-cache
	
	%对比\\ 和 \par 的用法有什么区别
	B. 在texlive安装路径bin/win32下,设置xelatex.exe以管理员身份启动设置xelatex.exe以管理员身份启动设置xelatex.exe以管理员身份启动。

	C. 启动TeXstudio时以管理员身份启动
	
	环境:win7 + texlive 2016 + TeXstudio
	
	问题:xelatex编译一个很短的文章时非常慢
	
	解决方法:
	
	A. 以管理员身份运行fc-cache
	
	B. 在texlive安装路径bin/win32下,设置xelatex.exe以管理员身份启动,设置xelatex.exe以管理员身份启动设置xelatex.exe以管理员身份启动设置xelatex.exe以管理员身份启动。
	
	C. 启动TeXstudio时以管理员身份启动
	\chapter{实验结果与分析}  % report和book类型的排版有chapter,article里没有
	\section{实验方法}
	\subsection{数据}
	\subsection{图表}
	\subsubsection{第一段}
	\subsubsection{第二段}
	\section{实验结果}
	\section{结论}
	\section{致谢}
\end{document}