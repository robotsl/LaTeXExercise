% 导言区
\documentclass{article} % article,book,report,letter 
\usepackage{ctex}   %命令行使用texdoc ctex 查看文档
\usepackage{float} 
\title{\heiti 我的第六个文档}
\author{\kaishu 廖兴滨}
\date{\today}
% 正文区
%与表格相关的文档可以使用texdoc booktab/longtab/tabu命令查看


\begin{document}  %一个tex文件中只能出现一个document
	\maketitle
	\begin{tabular}{l c c c r}
		姓名 & 语文 & 数学 & 外语 & 备注 \\
		张三 & 87 & 100 & 93 & 优秀 \\
		张三 & 87 & 100 & 93 & 优秀 \\
		张三 & 87 & 100 & 93 & 优秀 \\
	\end{tabular}
	
	\begin{tabular}{|| l || c || c || c || r ||}
		\hline \hline
		姓名 & 语文 & 数学 & 外语 & 备注 \\
		\hline \hline
		张三 & 87 & 100 & 93 & 优秀 \\
		\hline \hline
		张三 & 87 & 100 & 93 & 补考,另行通知 \\
		\hline \hline
		张三 & 87 & 100 & 93 & 优秀 \\
		\hline \hline
	\end{tabular}


	\begin{tabular}{| l | c | c | c | p{1.5cm} |} % 超过指定宽度会自动换行
		\hline 
		姓名 & 语文 & 数学 & 外语 & 备注 \\
		\hline 
		张三 & 87 & 100 & 93 & 优秀 \\
		\hline 
		张三 & 87 & 100 & 93 & 补考,另行通知 \\
		\hline 
		张三 & 87 & 100 & 93 & 优秀 \\
		\hline 
	\end{tabular}
	\begin{table}[H]
		%\renewcommand{\arraystretch}{1.3}
		\caption{本学期期末成绩表}
		\label{tab1}
		\centering
		\begin{tabular}{| l | c | c | c | p{2.5cm} |} % 超过指定宽度会自动换行
			\hline 
			姓名 & 语文 & 数学 & 外语 & 备注 \\
			\hline 
			张三 & 87 & 100 & 93 & 优秀 \\
			\hline 
			张三 & 87 & 100 & 93 & 补考,另行通知 \\
			\hline 
			张三 & 87 & 100 & 93 & 优秀 \\
			\hline 
		\end{tabular}
	\end{table}


\end{document}