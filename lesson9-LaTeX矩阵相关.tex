\documentclass{ctexart}

\usepackage{amsmath}

\newcommand{\adots}{\mathinner{\mkern2mu\raisebox{0.1em}{.}\mkern2mu\raisebox{0.4em}{.}\mkern2mu\raisebox{0.7em}{.}\mkern1mu}}
%感觉这里好难的感觉,定义一个反斜点
\begin{document}
	\[  %在这种写法中不可以出现空格否则会出错
	\begin{matrix}  % 普通得没有边界符号的矩阵
		0 & 1 \\   %用&分隔列,用\\分隔行
		1 & 0
	\end{matrix} \qquad
	\begin{pmatrix}  %矩阵两端加小括号
		1 & -i \\
		i & 0
	\end{pmatrix} \qquad
	\begin{bmatrix}  %矩阵两端加中括号
		0 & -1 \\
		1 & 0
	\end{bmatrix} \qquad
	\begin{Bmatrix}  %矩阵两端加花括号
		1 & 0 \\
		0 & -1
	\end{Bmatrix} \qquad
	\begin{vmatrix}  %矩阵两端加单竖线,注意此处与行列式不同
		a & b & e \\
		c & d & f
	\end{vmatrix} \qquad
	\begin{Vmatrix}  %矩阵两端加双竖线
		i & 0 \\
		0 & -i 
	\end{Vmatrix} \qquad
	\]
	
	\[
		A = \begin{pmatrix} % 使用pmatrix环境并且指定任意单个元素的形态
			a_{11} ^ 2 & a_{12} ^ 2 & a_{13} ^ 2 \\
			0 & a_{22} & a_{23} \\
			0 & 0 & a_{33}
			\end{pmatrix}
	\]
	
	\[
		%常用的省略号:\dots,\vdots,\ddots
		A = \begin{bmatrix}
			a_{11} & \dots & a_{1n} \\ % \dots表示横向的省略号
			\adots & \ddots & \vdots \\ % \ddots表示对角省略号,\vdots表示竖向省略号,而\adots表示自定义的反对角省略号,参考上面定义的newcommand写法
			0 &  & a_{nn}
			\end{bmatrix}_{n \times n} %表示nxn矩阵
	\]
	
	%分块矩阵,注意分块矩阵的写法,相当于矩阵内嵌套矩阵
	\[
		A = \begin{pmatrix}
			\begin{matrix}
				1 & 0 \\
				0 & 1
			\end{matrix} & \text{\LARGE 0} \\
			\text{\LARGE 0} & \begin{matrix}
				1 & 0 \\
				0 & -1
			\end{matrix}
		\end{pmatrix}
	\]
	
	%三角矩阵
	\[
		A = \begin{pmatrix}
			a_{11} & a_{12} & \cdots & a_{1n} \\
			& a_{12} & \cdots & a_{2n} \\
			& & \ddots & \vdots \\
			\multicolumn{2}{c}{\raisebox{1.3ex}[0pt]{\Huge 0}} & & a_{nn}
		\end{pmatrix}
	\]
	
	% 跨列的省略号
	\[
		A = \begin{pmatrix}
		1 & \frac 12 & \dots & \frac 1n \\
		\hdotsfor{4} \\
		m & \frac{m}{2} & \dots & \frac mn
		\end{pmatrix}
	\]
	
	
	% 行内小矩阵(smallmatrix)环境
	复数$ z = (x,y) $ 也可以用矩阵
	\begin{math}
		\left(  %需要手动加上左括号
		\begin{smallmatrix}
			x & -y \\ y & x
		\end{smallmatrix}
		\right) %需要手动加上右括号
	\end{math}来表示。
	
	% array环境(类似于表格环境tabular)
	\[
		A = \begin{array}{r|r} % {r|r}指定对应位置的元素对齐方式为居右
		\frac{1}{2} & 0 \\
		\hline  % 横线
		0 & -\frac abc \\ % 此处为省略写法,frac后面没有group符号,默认一个字符为一个占位元素
		\end{array}
	\]
	
	\[
		A = \begin{array}{c@{\hspace{-5pt}}l}
			\left(
				\begin{array}{ccc|ccc}
				a & \cdots & a & b & \cdots & b \\
				& \ddots & \vdots & \vdots & \adots \\
				& & a & b \\
				\hline
				& & & c & \cdots & c \\
				& & & \vdots & \ddots & \vdots \\
				\multicolumn{3}{c|}{\raisebox{2ex}[0pt]{\huge 0}} & c & \cdots & c
				\end{array}
			\right)
			&
			% 第一行
			\begin{array}{l}
				% \left.表示仅与\right\}配对,什么都不输出
				\left.\rule{0mm}{7mm}\right\}p\\ % 第一列
				\\
				\left.\rule{0mm}{7mm}\right\}q   % 第二列
			\end{array}
			\\[-5pt]
			% 第二行
			\begin{array}{cc}
				\underbrace{\rule{17mm}{0mm}}_m & % 第一列,\underbrac命令命令横向排版花括号,用下标的方式指定m标识符
				\underbrace{\rule{17mm}{0mm}}_n   %第二列
			\end{array}
		\end{array}
	\]
	
\end{document}