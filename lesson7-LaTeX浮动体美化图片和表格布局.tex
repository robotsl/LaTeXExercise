% 导言区
%\documentclass{article} % article,book,report,letter 
\documentclass{ctexart}
\usepackage{ctex}   %命令行使用texdoc ctex 查看文档
\usepackage{float} 
\usepackage{graphicx}
% 语法:\includegraphics[<选项>]{<文件名>}
% 格式:EPS,PDDF,PNG,JPEG,BMP
\graphicspath{{figures/},{pics/}}  %图片在当前目录下的 figure 目录
\title{\heiti 我的第六个文档}
\author{\kaishu 廖兴滨}
\date{\today}
% 正文区

% 浮动体
% 实现灵活分页(避免无法分割的内容产生的页面留白)
% 给图表添加标题
% 交叉引用

% figure环境(table环境类似)
% \begin{figure}[<允许位置>]
%  	 <任意内容>
% \end{figure}

% <允许位置>参数(默认tbp)
% h,此处(here)-代码所在的上下文位置
% t,页顶(top)-代码所在页面或之后页面的顶部
% b,页底(bottom)-代码所在页面或之后页面的底部
% p,独立一页(page)-浮动页面

\begin{document}  %一个tex文件中只能出现一个document
	\LaTeX{} 中的插图\ref{Russells-viper}: %设置引用
	
	\begin{figure}[H]
		\caption{这是一条大蛇}
		\label{Russells-viper}
		\centering
		\includegraphics[scale=0.3]{Russells-viper}
	\end{figure}
	
	
	在\LaTeX{} 中的表格也可以使用表\ref{tab_score}所示的表格:
	
	
	% 标题控制(caption、bicaption等宏包)
	% 并派与子图标(subcaption、subfig、floatrow等宏包)
	% 绕排(picinpar、wrapfig等宏包)
	\begin{table}[H]
		%\renewcommand{\arraystretch}{1.3}
		\caption{本学期期末成绩表}
		\label{tab_score}
		\centering
		\begin{tabular}{| l | c | c | c | p{2.5cm} |} % 超过指定宽度会自动换行
			\hline 
			姓名 & 语文 & 数学 & 外语 & 备注 \\
			\hline 
			张三 & 87 & 100 & 93 & 优秀 \\
			\hline 
			张三 & 87 & 100 & 93 & 补考,另行通知 \\
			\hline 
			张三 & 87 & 100 & 93 & 优秀 \\
			\hline 
		\end{tabular}
	\end{table}

	再看看一张曾经轰动韩国政界的新闻,见图\ref{test}.

	\begin{figure}[H]
		\caption{韩国新闻}
		\label{test}
		\centering
		\includegraphics[scale=0.7]{test}
	\end{figure}


\end{document}