\documentclass{article}

\usepackage{ctex}

\bibliographystyle{plain} % plain unsrt alpha abbrv

\begin{document}
	% 参考文献 的引用格式设置
	% 一次管理,一次使用
	% 参考文献各式:
	% \begin{thebibliography}{编号样本
	%	\bibitem[记号]{引用标志} 文献条目 1
	%	\bibitem[记号]{引用标志} 文献条目 2
	%	......
	% \end{thebibliography}
	% 其中文献条目包括:作者,题目,出版社,年代,版本,页码等。
	% 引用时可以采用:\cite{引用标志1,引用标志2,...}

	引用一篇文章\cite{article1} 引用一本书\cite{book1}

	\begin{thebibliography}{99}
		% 手动写参考文献
		\bibitem{article1}陈立辉,苏伟,蔡川,陈晓云.\emph{基于LaTex的Web数学公式提取方法研究}[J]. 计算机科学. 2014(06)
		\bibitem{book1}William H. Press,Saul A. Teukolsky,Willian T. Vetterling,Brain P. Flannery,\emph{Numerical Recipes 3rd Edition:The Art of Scientific Computing} Cambridge University Press, New York,2007.
	\end{thebibliography}
	
	%通过引用bibtex的方式引用文献,很简便,还要学会zotero和jabreference这样能提高生产力的插件和软件
	这是一个参考文献的引用: \cite{mittelbach2004} % 声明引用的article或者book等“被引关键字”(我也不知道这个叫啥,先这样叫吧)
	
	这是另一个参考文献的引用: \cite{OsbandDeep}
	
	这是另一个参考文献的引用: \cite{_expectimaxdouble_nodate}
	
	\bibliography{document11,dqn}  % 对于预先定义在bib文件中的内容,引用时需要将bib文件名在此处声明
	
	
	
\end{document}