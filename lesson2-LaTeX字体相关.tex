% 导言区
\documentclass[12pt]{article} % article,book,report,letter [10pt]表示十磅,只有10,11,12这三个
\usepackage{ctex}   %命令行使用texdoc ctex 查看文档

\newcommand{\myfont}{\textit{\textbf{Fancy Text}}} %这里相当于定义了一个宏

\title{\heiti 我的第二个文档}
\author{\kaishu 廖兴滨}
\date{\today}
% 正文区

\begin{document}  %一个tex文件中只能出现一个document
	\maketitle
	
	英文字体设置如下:
	
	% 字体族设置 (罗马字体、无衬线字体、打字机字体)
	\textrm{Roman Family} \textsf{Sans Serif Family} \texttt{Typewriter Family}
	{\rmfamily Roman Family} {\sffamily Sans Serif Family} {\ttfamily Typewriter Family}
	%可以用花括号声明作用域

	
	\sffamily who are you? what's your name? balabala
	
	\ttfamily Are you wise than others?
	
	% 字体设置系列设置(粗细、宽度)
	\textmd{Medium Series} \textbf{Boldface Series} %字体设置命令
	{\mdseries Medium Series} {\bfseries Boldface Series}  %字体设置声明
	
	%字体形状(直立、斜体、伪斜体、小型大写)
	\textup{Upright Shape} \textit{Italic Shape}
	\textsl{Slanted Shape} \textsc{Small Caps Shape}
	
	{\upshape Upright Shape} {\itshape Italic Shape}
	{\slshape Slanted Shape} {\scshape Small Caps Shape}
	
	中文字体设置如下:%需使用ctex宏包
	
	% 字体
	{\songti 宋体} \quad {\heiti 黑体} \quad {\fangsong 仿宋} \quad {\kaishu 楷书}
	中文的\textbf{粗体}与\textit{斜体}
	
	% 字体大小
	{\tiny Hello }\\
	{\scriptsize Hello}\\
	{\footnotesize Hello}\\
	{\small Hello}\\
	{\normalsize Hello}\\
	{\large Hello}\\
	{\Large Hello}\\
	{\LARGE Hello}\\
	{\huge Hello}\\
	{\Huge Hello}\\
	
	%中文字号设置命令
	\zihao{5}  你好!
	
	\myfont

	
\end{document}